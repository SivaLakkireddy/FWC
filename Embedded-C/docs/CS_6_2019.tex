\documentclass{article}
\usepackage{multimedia}
\usepackage[none]{hyphenat}
\usepackage{enumitem}
\usepackage{graphics}
\usepackage{graphicx}
\usepackage{tabularx}
\usepackage{listings}
\usepackage{ragged2e}
\usepackage{kvmap}
\usepackage{karnaugh-map}
\usepackage{multirow}
\usepackage[english]{babel}
\usepackage{caption}
\usepackage{tikz}
\usetikzlibrary{arrows,shapes,automata,petri,positioning,calc}

\title{VERIFICATION OF BOOLEAN IDENTITIES USING EMBEDDED-C}
\date{March 2023}
\author{Lakkireddy Veerasiva Reddy\\lakkireddyveerasivareddy@gmail.com - FWC22122\\IIT Hyderabad-Future Wireless Communication}
\date{}
\sloppy

\begin{document}

\maketitle
\tableofcontents
	\pagebreak
\section{PROBLEM}
\textbf{(GATE CS-2019)}\\
\textbf{Q.6} Which one the following is not a valid identity?
\begin{enumerate}[label=(\Alph*)]
	\item $ (x\oplus y)\oplus z = x\oplus (y\oplus z)$
	\item $ (x + y)\oplus z = x\oplus (y + z)$
	\item $ x\oplus y = x + y, if xy = 0$
	\item $ x\oplus y = (xy + x'y')'$
\end{enumerate}
\bigskip

\section{COMPONENTS}

\begin{table}[h]
\centering
\begin{tabular}{|c|c|c|}
  \hline
  \textbf{Symbol}&\textbf{Value}&\textbf{Description}\\
  \hline
  a & 8cm & BC\\
  \hline
  $\theta$ & $45^0$ & $\angle{BC}$ in $\Delta$ABC \\
  \hline
	k & 3.5 & AB-AC i.e(c-b)\\
  \hline
\end{tabular}

\caption{Components}
\label{table:Components}
\end{table}
\vspace{5mm}

\section{INTRODUCTION}
\paragraph{}
	An "identity" is merely a relationship that is always true, regardless of the values that any variables involved might take on; similar to laws or properties. Many of these can be analogous to normal multiplication and addition, particularly when the symbols {0,1} are used for {FALSE, TRUE}. 
\bigskip 

\section{TRUTH TABLE}
The Truth Table for the above identities is ass follows:
\begin{enumerate}[label=\textbf{(\Alph*})]
	\item \textbf{$  (x\oplus y)\oplus z = x\oplus(y\oplus z)$} \\
where $Y1=(x\oplus y)\oplus z,Y2=x\oplus(y\oplus z)$\\
\bigskip
\begin{table}[ht!]
	\centering
        \begin{tabular}{|p{5cm}|p{3cm}|p{2cm}|}
\hline
\multicolumn{3}{|c|}{COMPONENTS}\\
\hline
Component& Value& Quantity\\
\hline
Resistor& $\>$=220 Ohm& 1\\
\hline
Vaman& & 1\\
\hline
Arduino& UNO& 1\\
\hline
Seven Segent Display& Common Anode& 1\\
\hline
Decoder& 7447& 1\\
\hline
Flip Flop& 7474& 1\\
\hline
Jumper Wires&  & 20\\
\hline
Breadboard&  & 1\\
\hline
\end{tabular}

	\caption{Truth Table}
	\label{Table:Truth Table}
\end{table}
\bigskip
\bigskip
	
	\item \textbf{$(x+y)\oplus z=x\oplus(y+z)$}\\
where $Y1=(x+y)\oplus z, Y2=x\oplus(y+z)$\\
\bigskip
\begin{table}[ht!]
	\centering
       \begin{tabular}{|c|c|c|c|c|c|c|}                     
\hline counter & MSB & LSB & J & K & Q(t) & Q(t+1) \\                                                    
\hline 0 & 0 & 0 & 0 & 0 & 0 & 0 \\                  
\hline 1 & 0 & 1 & 1 & 1 & 0 & 1 \\                 
\hline 2 & 1 & 0 & 0 & 0 & 1 & 1 \\                 
\hline 3 & 1 & 1 & 1 & 1 & 1 & 0 \\                  
\hline                                               
\end{tabular}                                      

	\caption{Truth Table}
	\label{Table:Truth Table}
\end{table}
\bigskip

	\item \textbf{$x\oplus y = x + y, if xy = 0 $} \\
where $Y1=x\oplus y=x+y, if xy=0$\\
\bigskip
\begin{table}[ht!]
	\centering
     \begin{tabular}{ |c |c |c |c |c|}  
\hline 
\newline 
\textbf{x} & \textbf{y} & \textbf{Y1} & \textbf{Y2} & \textbf{F} \\
\hline 
 0 & 0 & 0 &0 &1 \\
 \hline
 0 & 1 & 1 &1 &1 \\
 \hline
 1 & 0 & 1 &1 &1 \\
 \hline
 \end{tabular}
 
	\caption{Truth Table} 
	\label{Table:Truth Table}
\end{table}
\bigskip

	\item \textbf{$ x\oplus y = (xy + x'y')'$}\\
		where $(xy+x'y')'=(x'+y')(x+y)$\\
		      $=x\oplus y$\\
The Truth Table for $x\oplus y$ is as follows:\\
\begin{table}[ht!]
	\centering
	\begin{tabular}{ |c |c |c |}  
\hline 
\newline
\textbf{x} & \textbf{y} & \textbf{$x\oplus y$} \\
\hline
0 & 0 & 0\\
\hline  
0 & 1 & 1\\
\hline
1 & 0 & 1\\
\hline
1 & 1 & 0\\
\hline 
\end{tabular}
 
	\caption{Truth Table} 
	 \label{Table:Truth Table}
\end{table}
\bigskip

\paragraph{}
	Here, Except \textbf{(B)} identity all other identies are valid according to the mentioned truth tables.
\end{enumerate}
\bigskip


\section{IMPLEMENTATION}
  \begin{table}[ht!]
    \centering	  
  \begin{tabularx}{0.46\textwidth} { 
  | >{\centering\arraybackslash}X 
  | >{\centering\arraybackslash}X 
  | >{\centering\arraybackslash}X  | }
\hline
\textbf{Arduino PIN} & \textbf{INPUT} & \textbf{OUTPUT} \\ 
\hline
\textbf 2 & x & \\
\hline
\textbf 3 & y & \\
\hline
\textbf 4 & z & \\
\hline
\textbf 13 & & F \\
\hline
\end{tabularx}

  \caption{Connections} 
  \label{Table:Connections}
  \end{table}
\bigskip

\subsection{PROCEDURE}
\begin{enumerate} 
\item Connect the circuit as per the above table.
\item Connect one end of the resistor to anode of LED and cathode of LED to ground.
\item Connect the output pin to LED.
\item Connect inputs to Vcc for logic 1, ground for logic 0.
\item Execute the circuit using the below code.
\end{enumerate}

\section{SOFTWARE}
  Now execute the following codes and upload in arduino to see the results.\\


\begin{tabularx}{1\textwidth} { 
  | >{\centering\arraybackslash}X |}
  \hline
  https://github.com/SivaLakkireddy/FWC/tree/main/Embedded-C/codes\\
  \hline
\end{tabularx}


\bibliographystyle{ieeetr}
\end{document}	
