\documentclass{article}
\usepackage[none]{hyphenat}
\usepackage{enumitem}
\usepackage{graphics}  
\usepackage{graphicx}
\usepackage{circuitikz}
\usetikzlibrary{circuits.logic.IEC,calc}
\usepackage{listings}
\usepackage{ragged2e}
\usepackage{tabularx}
\usepackage{float}
\usepackage{placeins}
\usepackage{booktabs}
\usepackage[utf8]{inputenc}
\usepackage{kvmap}
\usepackage{karnaugh-map}
\usepackage{multirow}
\usepackage[english]{babel}
\usepackage{caption}
\usepackage{tikz}
\usepackage{amsmath}
\usepackage{blindtext}
\usetikzlibrary{arrows,shapes,automata,petri,positioning,calc}

\title{IMPLEMENTATION OF BOOLEAN LOGIC BY USING ARDUINO WITH ASSEMBLY}
\author{LAKIREDDY VEERASIVA REDDY\\lakkireddyveerasivareddy@gmail.com\\FWC22122 IITH-Future Wireless Communications Assignment-2}
\lstset{    
       language=C++,
       basicstyle=\ttfamily\footnotsize,
       breaklines=true,
       frame=lines
       }
\begin{document}
\maketitle
\tableofcontents
\pagebreak

\section{Problem}                               
(GATE EC-2020)\\                                  
Q.No 10    The figure(Fig.\ref{fig:Circuit}) below shows a multiplexer where $S_1$ and $S_0$ are select lines, $I_0$ to $I_3$ are the input data lines, EN is the enable line, and $F(P,Q,R)$ is the output, $F$ is
\begin{figure}[!h]
\begin{center}
\begin{circuitikz}                                 
\node [muxdemux, muxdemux def={NL=5, NR=1, NB=2,w=4, Lh=4, Rh=4, square pins=1}](C) at (0,0){MUX};     
\node [right, font=\tiny] at (C.blpin 1){EN};      
\node [right, font=\tiny] at (C.blpin 2){$I_0$};   
\node [right, font=\tiny] at (C.blpin 3){$I_1$};   
\node [right, font=\tiny] at (C.blpin 4){$I_2$};   
\node [right, font=\tiny] at (C.blpin 5){$I_3$};   
\node [right=3mm, font=\tiny] at (C.brpin 1){F};   
\node [above, font=\tiny] at (C.bbpin 1){$S_1$};   
\node [above, font=\tiny] at (C.bbpin 2){$S_0$};   
\node [left=3mm, font=\tiny] at (C.blpin 1){0};    
\node [left=3mm, font=\tiny] at (C.blpin 2){R};    
\node [left=3mm, font=\tiny] at (C.blpin 3){0};    
\node [left=3mm, font=\tiny] at (C.blpin 4){R};    
\node [left=3mm, font=\tiny] at (C.blpin 5){1};    
\node [below=3mm, font=\tiny] at (C.bbpin 1){P};   
\node [below=3mm, font=\tiny] at (C.bbpin 2){Q};  
\node [ocirc, anchor=0](notin2) at (C.blpin 1){};  
%\draw (notin2.180) -- (C.bpin 1 -| extpin);       
\end{circuitikz}

\end{center}
\caption{Circuit}
\label{fig:Circuit}
\end{figure}

\begin{enumerate}
   \item $PQ +{Q^\prime} R$
   \item $P+Q {R^\prime}$
   \item $P{Q^\prime} R+{P^\prime}Q$
   \item ${Q^\prime} +PR$
\end{enumerate}

\section{Introduction}
A 4×1 multiplexer has four data inputs $I_3$, $I_2$, $I_1$ and $I_0$, two selection lines $S_1$ and $S_0$ and one output Y.One of these 4 inputs will be connected to the output based on the combination of inputs present at these two selection lines. 

\section{Components}
\begin{table}[!h]
\centering
\begin{tabular}{|c|c|c|}
  \hline
  \textbf{Symbol}&\textbf{Value}&\textbf{Description}\\
  \hline
  a & 8cm & BC\\
  \hline
  $\theta$ & $45^0$ & $\angle{BC}$ in $\Delta$ABC \\
  \hline
	k & 3.5 & AB-AC i.e(c-b)\\
  \hline
\end{tabular}

\caption{Components}
\label{table:Components}
\end{table}

\section{Implementation}
We know that the output of a multiplexer is given as: \\
$F=S_1^\prime S_0^\prime I_0+S_1^\prime S_0I_1+S_1S_0^\prime I_2+S_1S_0I_3$ \\ 
$F=P^\prime Q^\prime R+P^\prime Q(0)+PQ^\prime R+PQ(1)$ \\
$F=P^\prime Q^\prime R+PQ^\prime R+PQ$ \\
$F=Q^\prime R(P^\prime +R) +PQ$ \\
$F=Q^\prime R+PQ$     :$(P^\prime +R=1)$

\subsection{Truth Table}
\begin{table}[!h]
\centering
\begin{tabular}{|p{5cm}|p{3cm}|p{2cm}|}
\hline
\multicolumn{3}{|c|}{COMPONENTS}\\
\hline
Component& Value& Quantity\\
\hline
Resistor& $\>$=220 Ohm& 1\\
\hline
Vaman& & 1\\
\hline
Arduino& UNO& 1\\
\hline
Seven Segent Display& Common Anode& 1\\
\hline
Decoder& 7447& 1\\
\hline
Flip Flop& 7474& 1\\
\hline
Jumper Wires&  & 20\\
\hline
Breadboard&  & 1\\
\hline
\end{tabular}

\caption{Truth Table}
\label{table:Truth Table}
\end{table}

\subsection{K-map}
K-map follows as:
\begin{figure}[!h]
\begin{center}
\begin{karnaugh-map}[4][2][1][$QR$][$P$]
\maxterms{0,2,3,4}                       
\minterms{1,5,6,7}                                
\implicant{1}{5}
\implicant{7}{6}
\end{karnaugh-map}	
\end{center}
\caption{K-map}
\label{fig:k-map}
\end{figure}

\pagebreak
\subsection{Boolean Expression}
By Solving the above K-map, we get a boolean equation as: $F=PQ+{Q^\prime}R$

\section{IMPLEMENTATION}
\begin{table}[!h]   
\centering  
\begin{tabular}{|c|c|c|c|c|c|c|}                     
\hline counter & MSB & LSB & J & K & Q(t) & Q(t+1) \\                                                    
\hline 0 & 0 & 0 & 0 & 0 & 0 & 0 \\                  
\hline 1 & 0 & 1 & 1 & 1 & 0 & 1 \\                 
\hline 2 & 1 & 0 & 0 & 0 & 1 & 1 \\                 
\hline 3 & 1 & 1 & 1 & 1 & 1 & 0 \\                  
\hline                                               
\end{tabular}                                      
 
\caption{Connections}
\label{table:Connections} 
 \end{table}

\subsection{Procedure}
\begin{enumerate}
\item Connect the circuit as per the above table.
\item Connnect the one end of the resistor to anode of LED and cathode of LED to ground.
\item Connect the output pin to another end of resisor.
\item Connect inputs to Vcc for logic 1, ground for logic 0.
\item Execute the circuit using the below code.
   
\end{enumerate}

\section{Software}
   Now execute the following code  and upload  in arduino to see the results \\
   \vspace{5mm}
   \begin{tabularx}{1.1\textwidth} { 
  | >{\centering\arraybackslash}X |}
  \hline
  https://github.com/SivaLakkireddy/FWC/blob/main/Assembly/code/assembly.asm \\
  \hline
\end{tabularx}
\end{document}
